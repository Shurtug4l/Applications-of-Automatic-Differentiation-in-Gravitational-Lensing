\chapter{Summary and conclusions}
\label{chap:conclusions}

In this thesis, we have systematically explored the application of automatic differentiation (AD) and differentiable programming methods, utilizing the computational capabilities of PyTorch and TensorFlow, to address various aspects of gravitational lensing. This exploration was aimed at enhancing the optimization of parametric functions and models, which are critical for accurately modeling gravitational lensing phenomena. This concluding chapter aims to concisely summarize this work, emphasizing the integration of AD in gravitational lensing and detailing the specific topics addressed.

Automatic differentiation has emerged as a pivotal tool in our investigation, offering significant advantages over traditional numerical differentiation methods in terms of both accuracy and efficiency. AD's capability to compute derivatives of complex functions accurately and efficiently is particularly valuable in gravitational lensing, where precise parameter estimation is crucial. By leveraging the computational frameworks of PyTorch and TensorFlow, we have demonstrated the potential of AD to advance the analysis and interpretation of gravitational lensing phenomena, facilitating the development of more detailed and nuanced models.

In the first application, we tackled the challenge of microlensing light curve fitting by leveraging the standard Paczynski curve, focusing on the impact of key parameters such as the impact parameter, the maximum magnification time, and the Einstein radius crossing time on the microlensing event profile. Our approach began with simulating a microlensing event and introducing observational noise to mimic real data. Initial model fitting, using randomly assigned parameters, demonstrated the critical issue of degeneracy in microlensing analysis, where multiple parameter sets produce similar light curves, complicating parameter estimation.

By adjusting our model to explicitly include the Einstein crossing time, we improved the fit, illustrating the effectiveness of our method in mitigating degeneracy. In addition, Bayesian sampling provided deeper insight into the parameter space, confirming the degeneracy.

Secondly, we focused on parametric reconstruction of strong gravitational lensing using the Singular Isothermal Ellipsoid (SIE) model to analyze the mass distribution of lensing galaxies. Our methodology utilized the SIE model's parameters to simulate lensing effects and optimize these parameters based on the positions of multiple lensed images observed in astronomical data. We employed two optimization approaches: one that adjusted parameters based on their effects in the source plane and another that did so in the image plane, each aiming to reduce discrepancies between observed and predicted lensed images.

Through simulations that incorporate realistic observational conditions, we generated and optimized synthetic lensing scenarios. This process involved a detailed parameterization of both the lens and the source, which solved the lens equation numerically to predict the lensed image positions accurately. The optimization demonstrated the effectiveness of our methodology in aligning model predictions with observed data, as evidenced by significant improvements in model accuracy.

The results showcased the precision of the SIE model in replicating observed lensing phenomena, with optimized model parameters closely matching their intended values.

The final field of application focused on surface brightness and galaxy shapes: we have explored the critical role of accurately modeling surface brightness profiles for weak lensing analysis. This process is vital for measuring the shapes and orientations of galaxies, which are subtly altered by the gravitational lensing effects of foreground mass distributions. These alterations provide essential clues about the mass distribution of lensing objects, including elusive dark matter, and allow researchers to map the mass distribution of lensing structures, offering insights into dark matter and the universe's large-scale structure.

Our analysis focused on fitting a surface brightness profile using the Sérsic model, a versatile representation for various galaxy types. The fitting process involved simulating observational data with realistic noise and applying a Point Spread Function (PSF) to account for observational distortions. This step is crucial for separating the lensing signal from intrinsic galaxy alignments and observational effects, thereby enhancing the reliability of weak lensing measurements.

Through the simulation of a galaxy light profile, incorporating noise and PSF effects, we aimed to mimic the challenges encountered in real astronomical observations. The Sérsic profile was parameterized and fitted to this simulated data, optimizing its parameters to match the observed galaxy images. This process, facilitated by frameworks such as PyTorch and TensorFlow, involved the definition of a cost function and the employing of an optimizer for parameter updates, demonstrating practical applications of automatic differentiation.

This process highlights the importance of methodological advancements in fitting surface brightness profiles, not only for understanding individual galaxy structures, but also for leveraging weak lensing as a tool for probing the fundamental components and dynamics of the cosmos.

Looking ahead, AD promises to revolutionize gravitational lensing studies further by facilitating the development of sophisticated models for complex phenomena and enabling efficient analysis of vast astronomical datasets. Its integration into time-delay cosmography and non-parametric mass reconstruction could yield deeper insights into the universe's expansion and structure. Moreover, AD's role in processing data from upcoming large-scale surveys will be pivotal in uncovering new discoveries and enhancing our cosmological understanding.

In summary, AD's contribution to gravitational lensing fields is profound, offering a pathway to novel discoveries and a deeper understanding of the cosmos. Its continued development and application stand to unlock even greater potentials in astrophysical research.