\section{Strong lens parametric reconstruction}
\label{sec:strong_lens}

This section of the thesis delves into the intricacies of parametric strong lensing, focusing on the Singular Isothermal Ellipsoid (SIE) model, a widely adopted representation for the mass distribution of lensing galaxies. The SIE model provides a robust framework for simulating gravitational lenses, offering a simplified, yet effective, portrayal of their mass profiles.

Taking advantage of the positions of multiple images generated by the lensing effect, one can derive the parameters of the lensing system that produced them. The objective is to optimize the lens parameters, which dictate the lensing behavior and, consequently, the appearance and positions of the lensed images.

This optimization is pursued through two distinct methodologies, as discussed in \cref{sec:lens_reconstruction}. Starting in both techniques from the positions of the multiple images, the first method focuses on optimizing the lens parameters within the source plane, by mapping the positions of the images back in the source plane and minimizing the distance between the theoretical and inferred position of the source.

The second method shifts the focus to the lens plane by directly minimizing the distances between each observed and predicted image to optimize the lens parameters.

\subsection{Lens modelling}
\label{subsec:lens_modelling}
The first step, as always, is to produce a simulation of the lensing system and the main observables of this topic, the multiple images of the background source.

The process begins by defining the observational parameters: a grid of $100 \times 100 \,\SI{}{\pixel}$ is established, with a resolution of $\SI{0.03}{\arcsecond\per\pixel}$. A background noise per pixel of $rms = 0.5$ mimics real observational noise. Additionally, the final image is also convolved with a Gaussian PSF with FWHM of $\SI{0.1}{\arcsecond}$. Subsequently, the definition of the lens and source hyperparameters follows. Regarding the mass distribution of the lens, an SIE model is assumed in the center of the grid, at position $(0,0)$, characterized by a velocity dispersion of $\SI{200.0}{\kilo\meter\per\second}$, axial ratio of $0.3$ and position angle of $\SI{45}{\deg}$. Concerning the surface brightness distribution, a Sérsic profile is assumed for both objects: for the lens, an effective radius of $\SI{5.0}{\arcsecond}$ and a Sérsic index $n = 3.5$, while for the source these parameters are set to $\SI{0.3}{\arcsecond}$ and $4.0$. Moreover, the source is placed at position $(-0.2 \t_E, 0.1\t_E)$ (with $\t_E$ the lens Einstein radius\footnote{$\t_E = \SI{25.92e11}{} \bp{\dfrac{\s_v^2 D_{LS}}{c^2 D_S}} \, \SI{}{\arcsecond}\,$.}), and is characterized by an axial ratio and a position angle of $0.8$ and $\SI{60}{\deg}$, respectively. We then assume that the source galaxy hosts at its center a point-like source, \eg a quasar. Finally, it is assumed that the redshifts are $z_L = 0.3$ for the lens and $z_S = 1.5$ for the source. A summary of the lens and source parameters is presented in \cref{tab:parameters_summary}.

\begin{table}[t]
\centering
\caption{Summary of lens and source parameters.}
\label{tab:parameters_summary}
\begin{minipage}{0.5\textwidth}
\vspace{0pt}
\centering
\textbf{LENS}\par\medskip
\begin{tabular}{cc}
\toprule
Parameter & Value \\
\midrule
$z_L$ & \SI{0.3}{} \\
$s_v$ & \SI{200.0}{\kilo\meter\per\second} \\
$q$ & \SI{0.3}{} \\
$\varphi$ & \SI{0.7854}{\radian} \\
$f$ & \SI{1e3}{} \\
$R_e$ & \SI{5.0}{\arcsecond} \\
$n$ & \SI{3.5}{} \\
$x_{1c}$ & \SI{0.0}{\arcsecond} \\
$x_{2c}$ & \SI{0.0}{\arcsecond} \\
\bottomrule
\end{tabular}
\label{tab:lens_parameters}
\end{minipage}%
\begin{minipage}{0.5\textwidth}
\vspace{-14pt}
\centering
\textbf{SOURCE}\par\medskip
\begin{tabular}{cc}
\toprule
Parameter & Value \\
\midrule
$z_S$ & \SI{1.5}{} \\
$q$ & \SI{0.8}{} \\
$\varphi$ & \SI{1.0472}{\radian} \\
$f$ & \SI{4.5e3}{} \\
$R_e$ & \SI{0.3}{\arcsecond} \\
$n$ & \SI{4.0}{} \\
$y_{1c}$ & \SI{-0.1676}{\arcsecond} \\
$y_{2c}$ & \SI{0.0838}{\arcsecond} \\
\bottomrule
\end{tabular}
\label{tab:source_parameters}
\end{minipage}
\end{table}

By inserting these parameters into the model, it is possible to obtain all characteristics of the lens, but the first step required to simulate the strong lensing event is to solve the lens equation for the assigned position of the source, treated at this stage as point-like. This allows the positions of the multiple images produced to be assigned to the lens plane.


\subsection{Solving the lens equation}
\label{subsec:solving_lens_eq}
Given the complexity of the model, the best solution is to solve the lens equation numerically. To do this, the method proposed by \cite{kormann_isothermal_1994} has been adopted to solve the lens equation in the case of SIE models.
Starting with the two coordinates of the source (and considering $(x, \phi)$ as the radial and angular polar coordinates.),
\begin{subequations}
\begin{align}
    \label{eq:coords1}
    y_1 & = x_1 - \a_1(\vec{x}) \,,
    \\
    \label{eq:coords2}
    y_2 & = x_2 - \a_2(\vec{x}) \,,
\end{align}
\end{subequations}
and multiplying \cref{eq:coords1} by $\cos{\phi}$ and \cref{eq:coords2} by $\sin{\phi}$ we obtain
\begin{subequations}
\begin{align}
    \label{eq:coords11}
    y_1 \cos{\phi} & = x_1 \cos{\phi} - \a_1(\vec{x}) \cos{\phi} = x \cos^2{\phi} - \a (x, \phi) \cos^2{\phi} \,,
    \\
    \label{eq:coords22}
    y_2 \sin{\phi} & = x_2 \sin{\phi} - \a_2(\vec{x}) \sin{\phi} = x \sin^2{\phi} - \a (x, \phi) \sin^2{\phi} \,.
\end{align}
\end{subequations}

Recalling that $\a (x, \phi) = \Tilde{\Psi} (\phi)$, the angular part of the lensing potential, \cref{eq:coords11,eq:coords22} can be combined to calculate the distance from the lens center as a function of $\phi$:
\be
\label{eq:coord}
x (\phi) = y_1 \cos{\phi} + y_2 \sin{\phi} + \Tilde{\Psi} (\phi) \,.
\ee

By reinserting \cref{eq:coord} into the lens equation, and setting $q^\prime = \sqrt{1 - q^2}$, we obtain
\be
\label{eq:phi_func}
F (\phi) = \bs{y_1 + \frac{\sqrt{q}}{q^\prime} \arcsinh{\bp{\frac{q^\prime}{q} \cos{\phi}}}} \sin{\phi} - \bs{y_2 + \frac{\sqrt{q}}{q^\prime} \arcsin{\bp{q^\prime \sin{\phi}}}} \cos{\phi} = 0 \,.
\ee

The task of determining the images of a source located at the coordinates $(y1, y2)$ simplifies the identification of the roots of the function $F (\phi)$. After determining $\phi$, it can be applied to \cref{eq:coord} to calculate $x$. Analytical solutions are unattainable; thus, a numerical root-finding technique is required.

The algorithm adopted to find the roots of the function $F$ shares some similarities with Brent's method \citep{brent_algorithm_1971}. In particular, it uses a numerical technique based on the bisection method, which is a type of bracketing method. This method is used to find the roots of a continuous function. The approach is iterative and relies on the intermediate-value theorem. The bisection method provides a reliable and straightforward numerical approach to find roots with a guarantee of convergence, assuming that the initial interval indeed contains a root and the function does not change sign multiple times within the interval.

Thanks to this method, it is therefore possible to solve the lens equation to derive the positions of the multiple images and produce the simulated image of the strong lensing event.

The simulated image of the lens-source system is shown in \cref{fig:data_sl}.

\begin{figure}
  \begin{minipage}{\linewidth}
    \centering
    \subfloat[]{\includegraphics[width=0.6\linewidth, keepaspectratio]{img/chapter5/strong_lens/init_data.png}\label{fig:data_sl}}
  \end{minipage}
  \begin{minipage}{\linewidth}
    \centering
    \subfloat[]{\includegraphics[width=0.6\linewidth, keepaspectratio]{img/chapter5/strong_lens/init_data_super.png}\label{fig:data_sl_super}}
  \end{minipage}
  \caption[Simulated strong lensing event]{\protect\subref{fig:data_sl} Simulated image of a strong lensing event. The pixel scale is $\SI{0.03}{\arcsecond\per\pixel}$ and the image side length is $\SI{3}{\arcsecond}$. \protect\subref{fig:data_sl_super} The same simulated image, but with the tangential critical line (solid blue line), the cut (dashed red line) and the tangential caustic (solid red line) of the lens considered shown. The yellow dashed lines identify the lens center. The orange circles represent the positions of the multiple images obtained by solving the lens equation with the method discussed in \cref{subsec:solving_lens_eq}.}
  \label{fig:obs_sl}
\end{figure}


\subsection{Lens optimization}
Initially, the model of the lens galaxy is constrained solely by the positions of the quasar's multiple images. In practical terms, these positions, as determined from astronomical observations, are subject to a degree of imprecision. To simulate positional uncertainty, a minor scatter of $\SI{0.015}{\arcsecond}$ is introduced to the actual positions of the images.

To start the fitting process, we initialize the parameters of the SIE model to some values (see \cref{tab:parameters_sl}) and start by optimizing the lens parameter in the source plane.

\begin{table}[t]
\setlength{\extrarowheight}{2pt}
\setlength{\tabcolsep}{1pt}
\centering
\caption{Summary of the parameters before and after fitting the model using both source plane or image plane optimization.}
\label{tab:parameters_sl}
%\resizebox{0.2\linewidth}{!}{
\begin{tabular}{@{}c@{\hskip 10pt}c@{}@{\hskip 10pt}c@{}@{\hskip 10pt}c@{}@{\hskip 10pt}c@{}}
\toprule
Parameter         & True value                          & Initial value                         & Best-fit value                            & Best-fit value                            \\
                  &                                     &                                       & (Source plane opt.)                       & (Image plane opt.)                        \\ \midrule
$\s_v$            & \SI{200.0}{\kilo\meter\per\second}  & \SI{220.0}{\kilo\meter\per\second}    & \SI{200.1128}{\kilo\meter\per\second}     & \SI{200.9709}{\kilo\meter\per\second}     \\ 
$q$               & \SI{0.3}{}                          & \SI{0.7}{}                            & \SI{0.2990}{}                             & \SI{0.2952}{}                             \\
$\varphi$         & \SI{0.7854}{\radian}                & \SI{1.0472}{\radian}                  & \SI{0.7855}{\radian}                      & \SI{0.7853}{\radian}                      \\ \bottomrule
\end{tabular}
%}
\end{table}


\subsubsection{Source plane optimization}
\label{subsubsec:sl_source_plane}

The first step in the optimization process involves calculating the deflection angle at the positions of the multiple images derived by solving the lens equation. This allows them to be mapped onto the source plane. Since the initial model is generally not accurate, a ``source'' will be obtained for each of the multiple images of the quasar. It is then assumed that the best estimate for the unlensed position of the quasar is the average position of these sources. The multiple sources obtained and their average position are shown in \cref{fig:sp_pred}.

\begin{figure}
    \centering
    \includegraphics[width=0.8\linewidth, keepaspectratio]{img//chapter5//strong_lens/sp_pred.png}
    \caption[Initial sources predictions source plane optimization]{The multiple images of the source are mapped back to the source plane to multiple predicted sources (green triangles). Their mean position (purple triangle) is indicated. The solid blue curves represent the caustic and the cut of the lens model, respectively. The dashed blue curves give the true caustic and cut.}
    \label{fig:sp_pred}
\end{figure}

Subsequently, it is possible to define a cost function on the source plane that needs to be minimized. This is defined as the sum of the squared differences of the distances of the individual sources predicted from their average position.
By minimizing this cost function, it is possible to optimize the model parameters so that the multiple sources converge to a single position.

This is feasible, as always, through a training loop. After defining Adam as optimizer, training is started to minimize the cost function, which decreases from a starting value of $3.1\pwr{-1}$ to a final value of $1.1\pwr{-11}$ (see \cref{fig:sp_loss}).

In \cref{tab:parameters_sl} the best-fit results are reported. Furthermore, as can be seen in \cref{fig:best_fit_sp,fig:sp_best}, in both the source plane and the lens plane, the positions of the source and multiple images overlap. Additionally, from \cref{fig:sp_bestfit,fig:sp_bestfitimage} a clear coincidence of the caustic, cut, and critical line is visible.

\begin{figure}
  \begin{minipage}{0.49\linewidth}
    \centering
    \subfloat[]{\includegraphics[width=\linewidth, keepaspectratio]{img/chapter5/strong_lens/sp_loss.png}\label{fig:sp_loss}}
  \end{minipage}
  \begin{minipage}{0.49\linewidth}
    \centering
    \subfloat[]{\includegraphics[width=\linewidth, keepaspectratio]{img/chapter5/strong_lens/best_fit_sp.png}\label{fig:best_fit_sp}}
  \end{minipage}
  \caption[Best-fit model loss history and image positions for source plane optimization]{\protect\subref{fig:sp_loss} Loss history for source plane optimization and \protect\subref{fig:data_sl_super} best-fit image positions for the source plane optimization.}
  \label{fig:loss_best_images}
\end{figure}

\begin{figure}
  \begin{minipage}{\linewidth}
    \centering
    \subfloat[]{\includegraphics[width=0.6\linewidth, keepaspectratio]{img/chapter5/strong_lens/sp_best_fit.png}\label{fig:sp_bestfit}}
  \end{minipage}
  \begin{minipage}{\linewidth}
    \centering
    \subfloat[]{\includegraphics[width=0.6\linewidth, keepaspectratio]{img/chapter5/strong_lens/sp_best_fit_imageplane.png}\label{fig:sp_bestfitimage}}
  \end{minipage}
  \caption[Best-fit model source and image plane]{Best-fit model for the source plane optimization, visualized both \protect\subref{fig:sp_bestfit} in the source plane and \protect\subref{fig:data_sl_super} in the lens plane. On the source and lens planes are indicated the caustic, cut and critical line of the best-fit model and the true ones (which overlap).}
  \label{fig:sp_best}
\end{figure}

\subsubsection{Image plane optimization}
\label{subsubsec:image_plane_opt}

Regarding the optimization on the lens plane, the first part of the process is similar: the positions of the multiple images are mapped onto the source plane and their average position is calculated. However, unlike the process discussed previously, it is now necessary to remap the average position of the sources onto the lens plane by solving the lens equation. This allows the model-predicted positions of the multiple images to be obtained. These are shown in \cref{fig:ip_pred}.

\begin{figure}
  \begin{minipage}{\linewidth}
    \centering
    \subfloat[]{\includegraphics[width=0.6\linewidth, keepaspectratio]{img/chapter5/strong_lens/ip_pred.png}\label{fig:ip_pred}}
  \end{minipage}
  \begin{minipage}{\linewidth}
    \centering
    \subfloat[]{\includegraphics[width=0.6\linewidth, keepaspectratio]{img/chapter5/strong_lens/ip_dist.png}\label{fig:ip_dist}}
\end{minipage}
\caption[Multiple images prediction for image plane optimization]{\protect\subref{fig:ip_pred} Multiple images predicted by the model and true images superimposed on the light distribution of the image plane. \protect\subref{fig:ip_dist} Illustration in the lens plane of the multiple images predicted and observed, where each closest pair has been identified. Each black stick shows the distances between each quasar image and the closest model-predicted one. The tangential critical line of the model is represented by a solid blue curve, while the true one is shown as a dashed blue curve.}
  \label{fig:ip_preds}
\end{figure}

The process then proceeds as follows: for each observed image of the quasar, the closest predicted image from the model is identified. Then, the distance between each closest pair of predicted and observed images is measured. Such distances are shown as black sticks in \cref{fig:ip_dist}.

Once the closest pairs of images have been identified and their distance calculated, all that remains is to explore the parameter space in search of the combination of $\s_v$, $q$, and $\varphi$ that minimizes this distance, in order to make every image predicted by the model converge to the real image produced by the observed system. The cost function implemented to quantify this discrepancy between images is a $\chi^2$ in the lens plane that measures the sum of the absolute distances between each pair of images, divided by the observational error on each image position, as described in \cref{subsubsec:lens_plane_source_plane_optimization}.
With these settings, the minimization procedure starts with an initial loss value $\chi^2 = 5.05\pwr{3}$ and ends with a value of $\chi^2 = 1.56\pwr{-3}$, as reported in \cref{fig:ip_loss}. The values for the best-fit parameters are reported in \cref{tab:parameters_sl}. With these parameters, the model-predicted image positions match the observed image positions very well, as can be seen in \cref{fig:best_fit_ip}.

\begin{figure}
  \begin{minipage}{0.49\linewidth}
    \centering
    \subfloat[]{\includegraphics[width=\linewidth, keepaspectratio]{img/chapter5/strong_lens/ip_loss.png}\label{fig:ip_loss}}
  \end{minipage}
  \begin{minipage}{0.49\linewidth}
    \centering
    \subfloat[]{\includegraphics[width=\linewidth, keepaspectratio]{img/chapter5/strong_lens/ip_best.png}\label{fig:best_fit_ip}}
  \end{minipage}
  \caption[Best-fit model loss history and image positions for image plane optimization]{\protect\subref{fig:ip_loss} Loss history for image plane optimization and \protect\subref{fig:best_fit_ip} best-fit image positions for the image plane optimization.}
  \label{fig:loss_best_image_plane}
\end{figure}

Once the model has been fitted to the observational data and the best-fit parameters have been derived, these can be used to obtain information about the mass distribution of the lens.

In our particular case, the best-fit velocity dispersion results in an Einstein radius
\be
\label{eq:best_er}
\t_E = \SI{25.92e11}{} \bp{\dfrac{\s_v^2 D_{LS}}{c^2 D_S}} \SI{}{\arcsecond} = \SI{0.8379}{\arcsecond} = \SI{4.06e-6}{\radian} \,,
\ee
which, knowing the redshift $z_L = 0.3$ of the lens and so its angular diameter distance $D_L = \SI{9.5e5}{\kilo\parsec}$, can be converted into physical units:
\be
\label{eq:best_er_kpc}
R_E = \t_E D_L = \SI{3.86}{\kilo\parsec} \,.
\ee

Subsequently, we can compute the critical surface density $\S_{cr}$ from \cref{eq:2.18} as
\be
\label{eq:surfcrit}
\S_{cr} = \frac{c^2}{4 \pi G} \frac{D_S}{D_L D_{LS}} = \SI{2.41e9}{\msun\per\kilo\parsec\squared} \,,
\ee
allowing us to estimate the total mass enclosed inside the Einstein radius of the lens
\be
\label{eq:mass_er}
M(<R_E) = \pi R_E^2 \S_{cr} = \SI{1.13e11}{\msun} \,.
\ee