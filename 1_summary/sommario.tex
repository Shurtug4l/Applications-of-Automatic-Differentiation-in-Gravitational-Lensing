\selectlanguage{italian}
\chapter*{Sommario}
\addcontentsline{toc}{chapter}{Sommario}
\markboth{Sommario}{Sommario}
\pagestyle{empty}


Il lensing gravitazionale, una straordinaria conseguenza della teoria della relatività generale di Einstein, offre un'opportunità unica di esplorare le proprietà fondamentali dell'universo, studiando le distorsioni causate da oggetti massicci sul percorso dei raggi luminosi. Il lensing gravitazionale forte, in particolare, che si manifesta come immagini multiple altamente amplificate e distorte di sorgenti di background, offre preziose indicazioni sulla distribuzione della materia oscura e sulla formazione delle strutture cosmiche. Tuttavia, analizzare e modellare con precisione le lenti gravitazionali forti presenta varie difficoltà, a causa della natura complessa degli effetti di lensing e della vasta quantità di dati osservativi richiesti, oltre alle grandi risorse computazionali necessarie.
L'obiettivo principale di questo lavoro di tesi è quello di affrontare tali questioni sviluppando algoritmi Python avanzati per la precisa modellizzazione e analisi di lenti gravitazionali forti, fondati su tecniche di programmazione differenziabile, implementate utilizzando i framework PyTorch e TensorFlow.
Utilizzando modelli parametrici, questi algoritmi sfruttano la differenziazione automatica per retropropagare gli errori e calcolare i gradienti di una funzione di costo, facilitando l'ottimizzazione nello spazio dei parametri.
Attraverso l'implementazione in tali modelli parametrici, è possibile estrapolare le caratteristiche rilevanti e stimare i parametri chiave del sistema in esame. I modelli risultanti possono essere applicati a dati osservativi reali, migliorando la caratterizzazione e la classificazione delle lenti con maggiore precisione ed efficienza.
L'uso di PyTorch e TensorFlow nell'implementazione di questi algoritmi consente di utilizzare in modo efficiente le moderne risorse di calcolo, come le GPU, per gestire la complessità computazionale intrinseca all'analisi delle lenti forti. Inoltre, la flessibilità e l'estensibilità di questi framework consentono una perfetta integrazione con altri strumenti astrofisici e di calcolo, facilitando un'analisi completa e robusta.

La struttura di questa tesi è la seguente: nei \cref{chap:cosmology,chap:gravitational_lensing} viene presentata un'introduzione ai concetti principali della cosmologia e della teoria delle lenti gravitazionali, seguita da un'ampia descrizione dei più importanti modelli di lente nel \cref{chap:lens_models}. Il \Cref{chap:algorithms} descrive il paradigma della programmazione differenziabile, le sue caratteristiche e alcuni esempi di algoritmi per modellare e analizzare le lenti gravitazionali. Il \cref{chap:applications} si concentra specificamente sull'applicazione dei metodi di differenziazione automatica a problemi di ottimizzazione di microlensing e di lensing forte. Infine, viene presentato un esempio di analisi della brillanza superficiale per derivare la forma e l'ellitticità di una galassia, informazioni fondamentali per effettuare misurazioni di weak lensing.
