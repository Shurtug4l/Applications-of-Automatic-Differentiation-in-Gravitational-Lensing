\chapter*{Ringraziamenti}
\addcontentsline{toc}{chapter}{Ringraziamenti}
\markboth{Ringraziamenti}{Ringraziamenti}
\pagestyle{plain}

Nel momento in cui questo capitolo della mia vita si chiude e ne inizia uno nuovo, desidero esprimere la mia più profonda gratitudine a coloro che hanno reso possibile questo viaggio. La realizzazione di questa tesi magistrale non sarebbe stata possibile senza il sostegno, l'incoraggiamento e l'amore di molte persone straordinarie che mi hanno accompagnato lungo il percorso. È un privilegio avere l'opportunità di esprimere il mio sincero apprezzamento a queste persone speciali.

Non posso che iniziare senza dedicare il ringraziamento più speciale e profondo alla mia migliore amica, Federica. La nostra amicizia, iniziata per caso, si è trasformata in un legame indissolubile che ha arricchito e colorato ogni aspetto della mia vita, trasformando semplici momenti in ricordi preziosi che porterò sempre nel cuore. Federica, sei stata la mia roccia, la mia confidente, e la fonte di una forza incrollabile che mi ha permesso di navigare attraverso le difficoltà con determinazione e coraggio. Le tue parole di incoraggiamento, il tuo spirito indomito e la tua capacità di trovare la luce anche nei momenti più bui hanno acceso in me una scintilla di speranza e ottimismo, indispensabili per affrontare le sfide accademiche e personali.

Non meno importante è il ringraziamento che voglio estendere alla mia famiglia: ai miei genitori, Laura e Paolo, e a mia sorella, Giulia. Il vostro amore incondizionato, il vostro sostegno costante e la vostra fiducia in me hanno costituito il mio faro nella notte, guidandomi attraverso le incertezze e celebrando con me ogni traguardo raggiunto. La vostra presenza, anche se fisicamente distante, è stata una costante fonte di conforto e forza.

Un caloroso ringraziamento va esteso anche ai miei amici, pilastri della mia vita, fonte di gioia, ispirazione e sostegno. Edoardo, Mattia e Lorenzo, insieme abbiamo condiviso più di un decennio di risate, avventure e sfide. La distanza non ha fatto altro che rafforzare il legame che ci unisce, dimostrando che l'amicizia vera non conosce ostacoli.

Estendo la mia gratitudine ai miei compagni di università, Andrea, Leonardo e Chiara, con i quali ho condiviso non solo gli studi, ma anche un percorso di crescita personale. La nostra collaborazione, il sostegno reciproco e i momenti condivisi hanno arricchito significativamente la mia esperienza universitaria.

Un ringraziamento speciale va anche a un altro Andrea, una presenza costante nelle nostre ormai passate serate in aula studio. La tua compagnia ha reso quelle lunghe ore notturne non solo più produttive, ma anche piacevoli e ricche di amicizia.

Non posso trascurare di ringraziare il mio caro coinquilino Michele, la cui compagnia è sempre stata una fonte di allegria (e caffé).

In questo momento di riflessione, desidero anche esprimere la mia gratitudine a tutte quelle persone che, magari non citate esplicitamente, hanno arricchito il mio percorso con la loro presenza, il loro incoraggiamento e il loro affetto. Ogni piccolo gesto, ogni parola di sostegno, ogni momento condiviso ha contribuito a rendere questo viaggio indimenticabile.

Grazie di cuore a tutti voi per aver camminato al mio fianco in questa avventura, lasciando un'impronta indelebile nel mio cuore e nella mia anima.

\flushleft{Grazie a tutti.}
\flushright{Simone}